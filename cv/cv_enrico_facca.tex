%% start of file `template.tex'.
%% Copyright 2006-2013 Xavier Danaux (xdanaux@gmail.com).
%
% This work may be distributed and/or modified under the
% conditions of the LaTeX Project Public License version 1.3c,
% available at http://www.latex-project.org/lppl/.

\documentclass[11pt,a4paper,roman]{moderncv}
\usepackage[utf8]{inputenc}    % utf8 support       %!!!!!!!!!!!!!!!!!!!!
\usepackage[T1]{fontenc}       % code for pdf file  %!!!!!!!!!!!!!!!!!!!!

% possible options include font size ('10pt', '11pt' and '12pt'), 
% paper size ('a4paper', 'letterpaper', 'a5paper', 'legalpaper',
% 'executivepaper' and 'landscape') 
% and font family ('sans' and 'roman')
%\input{package.tex}
%\input{commands_mp.tex}


\usepackage[backend=biber,maxbibnames=10,defernumbers=true,sorting=ydnt]{biblatex}
\addbibresource{strings.bib} 
\addbibresource{pubblication_ef.bib} 

%\addbibresource{enrico_facca.bib} 
%\addbibresource{enrico_facca_sub.bib}

\defbibenvironment{mypubs}
 {\list
     {}
     {\setlength{\leftmargin}{\bibhang}%
      \setlength{\itemindent}{-\leftmargin}%
      \setlength{\itemsep}{\bibitemsep}%
      \setlength{\parsep}{\bibparsep}}}
  {\endlist}
  {\item}

% moderncv themes
\moderncvstyle{classic}      
% style options are 'casual' (default), 'classic', 'oldstyle' and 'banking'
\moderncvcolor{blue}
% color options 'blue' (default), 'orange', 'green', 'red', 'purple', 'grey' and 'black'
%\renewcommand{\familydefault}{\sfdefault}
% to set the default font; use '\sfdefault' for the default sans serif
% font,
% '\rmdefault' for the default roman one, or any tex font name
% \nopagenumbers{}                 
% uncomment to suppress automatic page numbering for CVs longer than one page

% character encoding
%\usepackage[utf8]{inputenc}
% if you are not using xelatex ou lualatex, replace by the encoding you are using
%\usepackage{CJKutf8}        
% if you need to use CJK to typeset your resume in Chinese, Japanese or Korean

% adjust the page margins
\usepackage[scale=0.75]{geometry}
%\usepackage[numbers]{natbib}
\usepackage{amssymb}
\usepackage{amsfonts}
%\usepackage{multibib}



%\newcites{other}{Cited Papers}
%\newcites{pub}{Published Journal Papers}
%\newcites{sub}{Submitted Papers}   \item
%\newcites{th}{Doctoral Dissertation}
%\newcites{prep}{Drafts}
%% \cventry{10-14-2020}{Presentation to the Vittorio Di Federico research group}   \item
%%         {Invited Research Seminary}
%%         {Emergence of branching structures via Optimal Transport and P-Laplacians}{Bologna (Italy)}{}
%% \item\Pres[Bologna (Italy)]{October 14th, 2020}
%%   {Presentation to the Vittorio Di Federico research group}
%%   {Emergence of branching structures via Optimal Transport and P-Laplacians}
%%   {Invited Research Seminary}
\newcommand{\Pres}[5]{\cventry{#2}{#5}{#3}{\emph{``#4''}}{#1}{}{}}
             
%%   \ifthenelse{\equal{#5}{}}
%%   {\emph{#3} ( #1 - #2 ) ``#4''Title:
%%   }
%%   {
%%     #5. ``#4'' \emph{#3} ( #1 - #2 ) 
%%   }
%% }
\newcommand{\Iedge}{e}
%\parindent=0cm
%\parskip=1em
\usepackage{comment}
\usepackage{booktabs}
\setlength\tabcolsep{6pt}
%\setlength{\hintscolumnwidth}{3cm} 
% if you want to change the width of the column with the dates
%\setlength{\makecvtitlenamewidth}{10cm}  
% for the 'classic' style, if you want to force the width
% allocated to your name and avoid line breaks. 
% be careful though, the length is normally calculated
% to avoid any overlap with your personal info;
%use this at your own typographical risks...




% personal data
\name{Enrico}{Facca}
\title{Academic Curriculum \tiny{(Updated at \today)}}
% optional, remove / comment the line if not wanted
\address{Piazza dei Cavalieri, 3}{56126 Pisa, Italy}
% optional, remove / comment the line if not wanted;
%the "postcode city" and "country" arguments can be omitted or provided empty
\phone[mobile]{+39~3334078760}
% optional, remove / comment the line if not wanted;
%the optional "type" of the phone can be "mobile" (default), "fixed" or "fax"
%\phone[fixed]{+2~(345)~678~901}
%\phone[fax]{+3~(456)~789~012}
\email{enrico.facca@sns.it enrico.facca@gmail.com} % optional, remove / comment the line if not wanted
%\homepage{http://www.math.unipd.it/\textasciitilde fpiazzon}
% optional, remove / comment the line if not wanted
%\social[linkedin]{john.doe}
% optional, remove / comment the line if not wanted
%\social[twitter]{jdoe}
% optional, remove / comment the line if not wanted
%\social[skype]{enrico.fh}
% optional, remove / comment the line if not wanted
\extrainfo{Date of birth: October 30th, 1989}
% optional, remove / comment the line if not wanted
\photo[64pt][0.4pt]{IMG_20191012}
% optional, remove / comment the line if not wanted;
%'64pt' is the height the picture must be resized to, 
%0.4pt is the thickness of the frame around it 
%(put it to 0pt for no frame) and 'picture' 
%is the name of the picture file
%\quote{Some quote} 
% optional, remove / comment the line if not wanted

% to show numerical labels in the bibliography
%(default is to show no labels);
% only useful if you make citations in your resume
\makeatletter
\renewcommand*{\bibliographyitemlabel}{\@biblabel{\arabic{enumiv}}}
\makeatother
\renewcommand*{\bibliographyitemlabel}{[\arabic{enumiv}]}
% CONSIDER REPLACING THE ABOVE BY THIS

% bibliography with mutiple entries
%\usepackage{multibib}
%\newcites{book,misc}{{Books},{Others}}
%----------------------------------------------------------------------------------
%            content
%----------------------------------------------------------------------------------
\begin{document}
\nocite{*}
%\begin{CJK*}{UTF8}{gbsn}                          % to typeset your resume in Chinese using CJK
%-----       resume       ---------------------------------------------------------
\makecvtitle
%Dichiarazione resa ai sensi degli artt. 46 e 47 DPR N. 445/2000
%\section{Personal info}
%\textbf{Date of birth}: October 30th, 1989 \\
%\textbf{Residence}: Via Citolo 10, Padova 35138, Italy.  


\section{Current Position}
\cventry{2019/Now}{Postdoc research fellowship}
{Centro di Ricerca Matematica Ennio De Giorgi - Scuola Normale Superiore, Pisa (Italy)}{Advisor: Michele Benzi} 
{}% (University of Padova)}
{}

\section{Positions Held}

\cventry{2018/2019}{Postdoc research fellowship}
{Padova University (Italy)}{} 
{Advisor: Mario Putti}% (University of Padova)}
{%Topic:
  %\begin{itemize}
  %\item 
  %Numerical solution and application of Branched Transport Problem
  %\item Sparse otpimization problems
  %\end{itemize}
}

\cventry{2018}{Scholarship researcher}
{Padova University (Italy)}{} 
{Advisor: Mario Putti}
{}

\section{Education}
\cventry{2018}{PhD in Mathematics}
{Padova University (Italy)}{}
{Supervisor: 
  Mario Putti,
  %(University of Padova),
  Co-advisor
  Franco Cardin 
  %(University of Padova)
  Thesis: \emph{Biologically inspired formulation of 
    Optimal Transport Problems}}
{}{}
\cventry{2014}{Master in Mathematics}
{Padova University (Italy)}{}
{Supervisor: Mario Putti, Co-advisor
  Franco Cardin.  Thesis: \emph{A biology-inspired model for the
Optimal Transport Problem}}
{}{}
% arguments 3 to 6 can be left empty
\cventry{2011}{Bachelor in Mathematics} {Padova University
  (Italy)}{}{Supervisor: Francesco Fass\`o. Thesis: \emph{Reduction of
    vector fields invariant under Lie group action}}{}


\section{Teaching and Tutoring experiences}
\cventry{2018/2019}{
  Numerical calculus, Matlab laboratory assistant
}{
  Aerospace Engineering Bachelor, University of Padova, 24 hours lectures}{}{}{}
\cventry{2016}{
  Pre-course for Physical-Mathematics Models
}{
  Mathematical Engineering Master, University of Padova, six hours lectures}{}{}{}

\cventry{2013/2014}{
  Tutor for Calculus and Linear Algebra
}{        
  Engineering Bachelor, 
  University of Padova, 24+24 exercise hours}{}{}{}

\section{Research interests}
\begin{itemize}
\item Numerical solution of Optimal Transport Problems:
  Monge-Kantorovich Problem, Congested and Branched Transport Problems
\item Biological and complex networks
\item Sparse Optimization, Basis Pursuit Problem
\end{itemize}

\section{Research activities summary}
In my PhD thesis, I introduced a dynamical formulation, called
Dynamical Monge-Kantorovich (DMK), of the Optimal Transport Problem
(OTP). The OTP is a type of optimization problem where the goal is to
determine the optimal reallocation of resources from one configuration
to another.  The definition of the DMK model equations and the develop
of methods for the numerical solution of the OTP have been summarized
in~\cite{Facca-et-al:2018,Facca-et-al-numeric:2020,Bergamaschi-et-al:2018}.
Recent advances in the theoretical consolidation of the DMK model are
described in~\cite{FaccaPiazzon:2019}, where we introduced a Gradient
Flow reinterpretation of the model. In~\cite{Facca-et-al-branch:2018}
we introduced an extension of DMK model addressing those problems
where concentration along the transport is either penalized or favored
is described.  More recently, I started exploring some adaptation of
the DMK model in finite dimensional frameworks, starting from the
ideas presented in~\cite{Facca-et-al-discrete:2018}.  Together with
Caterina De Bacco and her PhD students at Max Plank Institute in
T\"ubinghen, in~\cite{BaLe2020,lonardi2020optimal} we worked on the
application of a discrete version of the DMK model on graphs.  In
collaboration with the research group of Prof. Kurt Mehlhorn from Max
Plank Institute in Saarbr\"uchen,
in~\cite{Facca-et-al-nonuniform:2020} we studied a reformulation of
the discrete DMK model addressing the solution of Linear Programming
problem.  In~\cite{BoFa2020} we addressed the solution of the
multi-commodity Optimal Transport Problem i.e., the extension of the
OTP where different resources needs to be
reallocated. In~\cite{FaBe2020}, together with Prof. Michele Benzi
from Scuola Normale Superiore Pisa, we worked on the efficient
numerical solution of the OTP on graphs, combining backward Euler
time-stepping of the dynamical equations of the DMK model with inexact
Newton Method.

\section{List of Publications}
\begin{refsection}[pubblication_ef.bib,strings.bib]
\DeclareFieldFormat{labelnumberwidth}{}
\nocite{*}
\printbibliography[title={PhD Thesis}, type=misc, omitnumbers=true]%, resetnumbers=true] 
\end{refsection}

\newrefcontext[labelprefix=A]
\printbibliography[title={Peer-Reviewed Journal Articles}, type=article, resetnumbers=true]

%\newrefcontext[labelprefix=P]
%\makeatletter
%\begingroup
%\def\blx@prefixnumbers{P}
\newrefcontext[labelprefix=P]
\printbibliography[title={Preprints}, type=unpublished,resetnumbers=true]
%\endgroup



\section{List of Scientific Presentations}
\Pres{Bologna (Italy)}{10-14-2020}
  {Presentation to the Vittorio Di Federico research group}
  {Emergence of branching structures via Optimal Transport and P-Laplacians}
  {Invited Research Seminary}

\Pres{Egmood aan Zee (Netherland)}
  {10-02-2019} {European Numerical
    Mathematics and Advanced Applications Conference 2019}
  {Optimal Transport Tools on Surface}
  {Contributed Talk}

\Pres{Cortona (Italy)} 
  {06-26-2019} 
  {People in
    Optimal Transportation and Applications - Incontri Indam
    2019
  } 
  {A nature inspired optimization tool}
  {Contributed Talk}

\Pres{Houston - Texas, USA }
  {04-14-2019}
  {SIAM Conference on Mathematical Computational Issues in the Geosciences}
  {Plant Root Modeling via Optimal Transport}
  {Contributed Talk}

\Pres{Houston - Texas, USA }
  {04-12-2019}
  {SIAM Conference on Mathematical Computational Issues in the Geosciences}
  {Numerical Solution of $L^1$-Optimal Transport Problem}
  {Contributed Talk}

\Pres{Pisa, Italy}
  {11-15-2018}
  {Optimal Transportation and Applications }
  {Biologically inspired deduction of Optimal
    Transport Problems}
  {Invited Presentation}
  
\Pres{Bergen, Norway}
  {09-06-2018}
  {Presentation to porous media research group} 
  {Biologically inspired formulation of Optimal
    Transportation Problems}
  {Research Seminar}


\Pres{Roma, Italy}
  {07-06-2018}
  {SIMAI 2018}
  {Biologically inspired formulation of Optimal
      Transportation Problems}
  {Contributed Talk}

\Pres{Saint Malom, France}
  {06-04-2018}
  {Computational Methods in Water Resources XXIII}
  {Plant root dynamics via Optimal Transport}
  {Contributed Talk}


\Pres{Bonn, Germany } 
  {04-05-2018}
  {Terrestrial Systems Research: Monitoring, Prediction and
    High Performance Computing} 
  {Hydrological networks as
    optimal transport structures} 
  {Contributed Talk}


\Pres{Erlangen, Germany}
  {09-11-2017}
  {SIAM Conference on Mathematical 
    and Computational Issues in the Geo-sciences}
  {Biologically inspired formulation of Optimal
    Transportation Problems}
  {Contributed Talk}

  
 \Pres{Lausanne, Swizerland}
   {04-27-2017}
   {Presentation to the Fabio Nobile research group}
   {Biologically inspired formulation of Optimal
     Transportation Problems}
   {Research Seminar}
      

\Pres{Bressanone, Italy} 
  {12-18-2014}
  {Current Problems in fluid-dynamics 
    and non equilibrium thermodynamics}
  {Biologically inspired formulation of Optimal Transportation
    Problem}
  {Contributed Talk}%% 
%% \end{enumerate}

% \section{Research achievements}
% I am author and co-author of two accepted journal paper
% and one submitted paper (see \emph{List of Publications} below).
% I attended to several international conferences
% where I presented my results
% (see \emph{List of Scientific Presentations} below).

% My research interests range from the numerical analysis to
% the solution of optimization problems, mainly focused on
% the \emph{Optimal Transport Problem} (OTP), 
% an area of mathematics studies 
% the least-cost strategies to reallocate 
% ``resources''. In particular, my research interests 
% are focused in the study of following formulation of the \OTP.
% Given two balanced densities $\Source$ and $\Sink$ defined in 
% $\Domain \subset \REAL^{\Dim}$ find the vector field
% $\Vel \Domain \mapsto \REAL^{\Dim}$ solving
% \begin{equation}
%   \min_{\Vel}
%   \left\{
%     \int_{\Domain} |\Vel|^{\PPP} \ : \ \Div(\Vel(x))=\Source(x)-\Sink(x)  
%   \right\}
% \end{equation}
% with $\PPP \in ]0,2[$, 
% (the case $\PPP <1$ requires a proper chracterization of the 
% above integral).

% The approach I proposed (firstly in my master degree thesis
% and then in my PhD studies) is based on the approximation of
% the  optimal vector field as the long-time limit 
% of $\Vel(t,x) = -\TDens(t,x) \Grad \Pot(t,x)$
% where the pair
% $(\Tdens,\Pot):([0,+\infty[,\Domain)\mapsto (\REAL^+,\REAL)$
% solves the following equations 
% \begin{equation}
%   \label{eq-sys}
%   \begin{aligned}
%     %\label{eq-sys-div}
%     &-\Div\left(\Tdens(t,x)\Grad\Pot(t,x)\right) 
%     =
%     \Forcing(x)=\Source(x)-\Sink(x) 
%     \\ 
%     %\label{eq-sys-1-dyn}
%     & \Dt{\Tdens}(t) =
%     \left[\Tdens(t,x)|\Grad\Pot(t,x)|\right]^{\Pflux}
%     -
%     \Tdens(t)
%     \\
%     %\label{eq-sys-1-bc} 	
%     &\Tdens(0) =\Tdens_0(x)> 0 
%   \end{aligned}
% \end{equation}
% completed with zero Neumann-Boundary conditions.
% % For the $\Pflux=1$, I conjectured
% % the equivalence between long time-limit of the above
% % system of equations and the solution of  the
% % Monge Kantorovich Equations (\MKEQS), a PDE formulation
% % of the \OTP\ introduced in~\citeother{Evans-Gangbo:1999}.
% % During my PhD studies, I carried out a deeper
% % investigation of the conjecture.
% Although a complete proof is still missing,
% somseveralthe numerical evidence shows that the numerical solution 
% of the \MKEQS\ via the proposed model 
% is very efficient, well-defined and robust,
% using simple discretization schemes. 
% The results obtained are summarized in~\citepub{Facca-et-al:2018} 
% and in~\citesub{Facca-et-al-numeric:2017}.


% The simple modifications of the proposed model, with 
% general exponent $\Pflux>0$, has been introduced during
% my PhD studies. For $0<\Pflux<1$ the above system of equations
% leads to the solution of the \CTP, with
% $\OptPot=\lim_{t\Tendsto +\infty} \Pot(t,\cdot)$ solving
% the $\Plapl$-Laplancian equations 
% with $\Plapl=(2-\Pflux)/(1-\Pflux)$.
% For $\Pflux>1$, numerical simulations shows that the long
% time limit of equation~\eqref{eq-sys} 
% leads to the formation of singular and
% fractal-like structures typically emerging from BT problems.
% Despite the exact relationship between the steady state
% of equation~\eqref{eq-sys} and the \BTP\ 
% is still missing, again partial 
% mathematical results and strong numerical evidence support
% our claim that these structures are solutions of a \BTP,
% possibly non-standard.
 
% I plan to submit a paper
% summarizing all these results in few months.

% For example, the problems where we want to 
% penalize aggregation along transport is called 
% \emph{Congested Transport Problem}(\CTP), and it finds
% application in study of the crown motion or urban traffic
% planning. On the contrary, \emph{Branched Transport
%   Problem}(\BTP) studies those problems were mass
% concentration is favored, creating fractal-like and ramified
% transport structures that are ubiquitous in nature, like the
% plant roots, blood vessels, and river networks.

% The model for $\Pflux>1$ finds different application
% in the study of natural transport network like
% plant roots, blood vessels, and river networks, 
% both in the description of the formation of such structure
% and in the understanding of the leading principle shaping them.

% The numerical solution via the proposed model has many
% strength point, in particular its
% simplicity.
% In fact, the heaviest computational cost is payed solving 
% sequence of sparse semi-positive definite linear system
% were there is still huge space of improvement, 
% in particular in the identification of 
% ad-hoc preconditioning stratergies.
% In~\citepub{Bergamaschi-et-al:2018} we investigate
% how to incorporate partial approximated spectral information
% into several preconditioning strategies.

  
% My research interests range from the 
% theoretical to the numerical analysis of the 
% \emph{Optimal Transport Problem} (OTP),
% in particular of the Monge-Kantorovich (MK) equations,
% Congested and Branched Transport Problem 
% (\CTP,\BTP), all studying 
% least-cost strategies to reallocate ``resources''.
% The MK equations is a PDE based formulation
% of the \OTP, while the \CTP\ and the \BTP\
% study those phenomena where  mass
% aggregation is either penalized or favored.
% In particular, the \BTP\ finds many applications
% in the study of natural network since
% like the plant roots, blood
% vessels, and river basins.


% % contribuition
% My contributions in these topics are connected to
% a model ( introduced in my master degree and main topic of
% my PhD studies~\citeth{Facca:2018})
% that couples a diffusion equation 
% with an ODE imposing a transient
% dynamics to the diffusion coefficient.
% I conjectured this system to convergence 
% toward a steady state configuration 
% and to solve OT problems.
% Although a complete proof is still missing,
% several the numerical evidence shows 
% that the numerical solution 
% of the OT problems via the proposed model 
% is very efficient, well-defined and robust,
% using simple discretization schemes
% (all these results are collected 
% in~\citepub{
%   Facca-et-al:2018}~\citesub{
%   Facca-et-al-numeric:2018,
%   Facca-et-al-branch:2018}).
% Simplicity is one of the many strength point of the model.
% In fact, the heaviest computational cost is payed solving 
% sequence of sparse semi-positive definite linear system
% where there is still huge space of improvement, 
% in particular in the identification of 
% ad-hoc preconditioning strategies.
% In~\citepub{Bergamaschi-et-al:2018} I worked investigating
% how to incorporate partial approximated spectral information
% into several preconditioning strategies.

% % current and future works
% A similar linear algebra problem, now
% part of my post-Doc research,  
% arises from the backward Euler time-stepping
% of the model equations, 
% where non-linearity is solved via Newton method.
% The problem requires the solution of 
% a sequence of linear systems 
% in the form
% \begin{equation*}
% \begin{pmatrix}
%     \Matr{A}
%     &
%     \Matr{B}^T
%     \\
%     -\Deltat 
%     \Matr{D_1}
%     \Matr{B}
%     &
%     \quad \Matr{I}-\Deltat\Matr{D_2}
%   \end{pmatrix}
%   \begin{pmatrix}
%     \Vect{x}_1
%     \\
%     \Vect{x}_2
%   \end{pmatrix}
%   =
%   \begin{pmatrix}
%     \Vect{b}_1
%     \\
%     \Vect{b}_2
%   \end{pmatrix}
% \end{equation*}
% where $\Deltat$ is the time-step,
% $\Matr{A}$ and $\Matr{B}$ are sparse matrices,
% the first semi-positive definitive, the second rectangular,
% while $\Matr{D_1}$ and $\Matr{D_2}$ are diagonal matrices.
% An efficient preconditioning strategies will
% give enormous speed-up in the solution of OT problems 
% since preliminary results 
% already shows that convergence to steady state 
% is achieved after 30 time-iterations, requiring 
% between 4 and 6 Newton iterations each.
 

% The same linear algebra problem arises from 
% a discrete version of the model 
% that can be applied in the finding sparse
% solution of undetermined linear system,
% minimizing the $l_1$-norm of the solution
% (problem know also as Basis Pursuit). 


\section{Master Thesis Co-supervision}
 Co-advisor with Mario Putti of the master degree thesis in
 Mathematics and Mathematical Engineering of Andrea Pinto
 (2015), Enrico Cortese (2017), Claudia Dario (2017),
 Riccardo Tosi (2018), Luca Berti (2018), Nicola Segala (2019).

\section{Visiting periods}
\cventry{2019}{Max Planck Institute Saarbr\"ucken, Germany}{ One week
  working on Polycephalum model for Basis Pursuit Problem with the
  research group of Kurt Mehlhorn}{}{}{}

\cventry{2018}{Bergen University, Norway}{One week working on the
  application of the Branched Transport Problem to the study of blood
  vessel in the brain with the research group of Jan Martin
  Nordbotten}{}{}{}

\cventry{2016}{Orsay - Paris Sur University,
  France}{Two weeks working on my doctoral dissertation with with
  Filippo Santambrogio}{}{}{}

\cventry{2015-2016}{Friedrich-Alexander-Universit\"at
  Erlangen-N\"urnberg Germany}{Seven months working on my doctoral
  dissertation with Peter Knabner, Aldo Pratelli, and Sara
  Daneri}{}{}{}

  

\section{Organizing activities}
 \begin{itemize}
 \item Organizer of a two-days workshop
   \emph{``Seminari Padovani di Analisi Numerica 2018''} 
   May 2018, Padova.
 \end{itemize}


\section{Computer skills}
  \begin{itemize}
    \item Advanced knowledge with Unix system
    \item Expert with Fortran, including Object-Oriented Fortran 2003-2008
    \item Advanced knowledge with Python, Matlab
  \end{itemize}

\section{Languages}
\cvitemwithcomment{Italian}{Mother tongue}{}

\cvitemwithcomment{}{%
  \centering
  \begin{tabular}{cccc}
    \toprule
    \multicolumn{2}{c}{Understanding} 
    & \multicolumn{1}{c}{Speaking}
    & Writing \\ \midrule
    Listening         & Reading        &  -  & -       
    %\\ B2 & B2  & B2  & B2  & B2      \\ \bottomrule
  \end{tabular}%
}{}
\cvitemwithcomment{English}{%
  \centering
  \begin{tabular}{*{4}{c}}
    \toprule
    % \multicolumn{2}{c}{Understanding} 
    % & \multicolumn{2}{c}{Speaking}
    % & Writing \\ \midrule
    % & & Spoken & Spoken & \\
    % Listening         & Reading        &  interaction &  production &
    % -       
    % \\ 
    \hspace{0.43cm}  B2 & 
    \hspace{0.88cm}  C1 \   & 
    \hspace{0.8cm}  B2  & 
    \hspace{0.8cm}   B2 \hspace{0.32cm}  
    \\ 
    \bottomrule
  \end{tabular}
}{}
\cvitemwithcomment{Spanish}{%
  \centering
  \begin{tabular}{*{4}{c}}
    %\toprule
    % \multicolumn{2}{c}{Understanding} 
    % & \multicolumn{2}{c}{Speaking}
    % & Writing \\ \midrule
    % & & Spoken & Spoken & \\
    % Listening         & Reading        &  interaction &  production &
    % -       
    % \\ 
    \hspace{0.43cm}  B2 & 
    \hspace{0.88cm}  C1 \   & 
    \hspace{0.8cm}  B2  & 
    \hspace{0.8cm}   B2 \hspace{0.32cm}  
    \\ 
    \bottomrule
  \end{tabular}
}{}

\section{References}
\emph{For a reference letter please contact the following persons.}

\begin{cvcolumns}
  \cvcolumn{}{
    \begin{itemize}
    \item Michele Benzi (Scuola Normale Superiore, Italy) 
    \item Giuseppe Buttazzo (University of Pisa, Italy)
    \item Franco Cardin (University of Padova, Italy) 
    \item Malgorzata Peszynska (Oregon State University, USA) 
    \item Mario Putti (University of Padova, Italy) 
    \end{itemize}
  }
\end{cvcolumns}


\newpage




   %\bibliographystyleother{plainyr-rev}
   %\bibliographyother{biblio}


\end{document}


% end of file `template.tex'.
